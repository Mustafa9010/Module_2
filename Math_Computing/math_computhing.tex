\documentclass[11pt, a4paper]{article}
\usepackage{sectsty}
\usepackage{graphicx}
\usepackage{amsmath}
\usepackage{amssymb}
\usepackage{setspace}
\usepackage{tasks}
\usepackage{graphicx}
\usepackage{float}
\usepackage{comment}
\usepackage{listings}
\usepackage[utf8]{inputenc}
\usepackage{amsfonts}
\usepackage{gensymb}
\usepackage{multicol}
\usepackage{tabularx}
\usepackage{tikz}
\newcommand{\myvec}[1]{\ensuremath{\begin{pmatrix}#1\end{pmatrix}}}
\let\vec\mathbf
\newcommand{\abs}[1]{\left| #1 \right|}
\newcommand{\mydet}[1]{\ensuremath{\begin{vmatrix}#1\end{vmatrix}}}
\providecommand{\brak}[1]{\ensuremath{\left(#1\right)}}
\providecommand{\lbrak}[1]{\ensuremath{\left(#1\right.}}
\providecommand{\rbrak}[1]{\ensuremath{\left.#1\right)}}
\providecommand{\cbrak}[1]{\ensuremath{\left\{#1\right\}}}
\providecommand{\sbrak}[1]{\ensuremath{{}\left[#1\right]}}
\providecommand{\norm}[1]{\left\lVert#1\right\rVert}
\providecommand{\abs}[1]{\left\vert#1\right\vert}
\usepackage[left=1in, right=0.5in, top=1in, bottom=1in]{geometry}
\title{ Math computing}
\author{ Mustafa}
\date{\today}
\let\cleardoublepage\clearpage
\begin{document}
\section*{NCERT CLASS 12}
\subsection*{CHAPTER 10 : EXERCISE 5.13}
\begin{enumerate}
\item The Scalar product of the Vector $(\hat{i}+\hat{j}+\hat{k})$ with a Unit Vector along the Sum of Vector $(2\hat{i}+4\hat{j}-5\hat{k})$ and $(\lambda \hat{i}+2\hat{j}+3\hat{k})$ is Equal to One.Find the Value of "$\lambda$". 
\\

\textbf{CONSTRUCTION STEPS :}
\begin{enumerate}
    \item Let us consider the given Data,and find $\lambda$ Value
    \begin{align}
        \vec{A}=\myvec{1\\2\\ \lambda};\vec{B}=\myvec{1\\4\\2};\vec{C}=\myvec{1\\-5\\3}
    \end{align}
       \item We know that ,\\
       \begin{align}
           \vec{A}^{\top}\brak{\frac{\brak{\vec{B}+\Vec{C}}}{\norm {\vec{B}+\Vec{C}}}}=1\label{eq:Eqt1}
       \end{align}
       \item since from given data,we get
   \begin{align}
    \implies  \Vec{A}^{\top}=\myvec{1&2&\lambda}\\
            \implies   \brak {\Vec{B}+\Vec{C}}=\myvec{2\\-1\\5}\\
          \implies  \brak{\Vec{B}+\Vec{C}}^{\top}=\myvec{2&-1&5}
          \end{align}
          \item For $\norm{\Vec{B}+\Vec{C}}$,
          \begin{align}
              \norm{\Vec{B}+\Vec{C}}=\sqrt{\brak{\Vec{B}+\Vec{C}}^{\top}\cdot\brak{\Vec{B}+\Vec{C}}}\\
              \norm{\Vec{B}+\Vec{C}}=\sqrt{\myvec{2&-1&5}\cdot\myvec{2\\-1\\5}}\\
             \implies   \norm{\Vec{B}+\Vec{C}}=\sqrt{30}
          \end{align}
          \item Substitute the Values in Equation\eqref{eq:Eqt1},then we get
          \begin{align}   
                      \myvec{1&2& \lambda}\myvec{\frac{2}{\sqrt{30}} \\ \frac{-1}{\sqrt{30}} \\ \frac{5}{\sqrt{30}}}=1\\
                      \frac{5\lambda}{\sqrt{30}}=1\\       
                     \implies \lambda=1.093151\\
                      \implies \lambda \approx 1\\
          \end{align}
       Hence  $\therefore \lambda$=1;
      \end{enumerate}
   \end{enumerate}
\end{document}

