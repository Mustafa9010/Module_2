\documentclass[11pt, a4paper]{article}
\usepackage{sectsty}
\usepackage{graphicx}
\usepackage{amsmath}
\usepackage{amssymb}
\usepackage{setspace}
\usepackage{tasks}
\usepackage{graphicx}
\usepackage{float}
\usepackage{comment}
\usepackage{listings}
\usepackage[utf8]{inputenc}
\usepackage{amsfonts}
\usepackage{gensymb}
\usepackage{multicol}
\usepackage{tabularx}
\usepackage{tikz}
\newcommand{\myvec}[1]{\ensuremath{\begin{pmatrix}#1\end{pmatrix}}}
\let\vec\mathbf
\newcommand{\abs}[1]{\left| #1 \right|}
\newcommand{\mydet}[1]{\ensuremath{\begin{vmatrix}#1\end{vmatrix}}}
\providecommand{\brak}[1]{\ensuremath{\left(#1\right)}}
\providecommand{\lbrak}[1]{\ensuremath{\left(#1\right.}}
\providecommand{\rbrak}[1]{\ensuremath{\left.#1\right)}}
\providecommand{\cbrak}[1]{\ensuremath{\left\{#1\right\}}}
\providecommand{\sbrak}[1]{\ensuremath{{}\left[#1\right]}}
\providecommand{\norm}[1]{\left\lVert#1\right\rVert}
\providecommand{\abs}[1]{\left\vert#1\right\vert}
\usepackage[left=1in, right=0.5in, top=1in, bottom=1in]{geometry}
\title{ Math computing}
\author{ Mustafa}
\date{\today}
\let\cleardoublepage\clearpage
\begin{document}
\section*{NCERT CLASS 12}
\subsection*{CHAPTER 10 : EXERCISE 5.13}
\begin{enumerate}
\item\textbf{}The scalar product of the vector $\hat{i}+\hat{j}+\hat{k}$ with a unit vector along the sum of vectors $2\hat{i}+4\hat{j}-5\hat{k}$ and $\lambda\hat{i}+2\hat{j}+3\hat{k}$ is equal to one, Find the value of $\lambda$.
\\\\
\textbf{Generalized Construction:}\\
We now that \\
\begin{align}
   &\implies \vec{A}^\top = \frac{\brak{\vec{B}+\vec{C}}}{\norm{\vec{B}+\vec{C}}}\\
       &\implies \vec{A}^\top \brak{\vec{B}+\vec{C}}=\norm{\vec{B}+\vec{C}} \label{eq:Eqat2}\\
       &\implies \vec{C}=\lambda\vec{e}_1+\vec{D}\label{eq:EQT-C}
    \end{align}
    were,
    \begin{align}
       &\implies \norm{\vec{B}+\vec{C}}= \sqrt{\brak{\vec{B}+\vec{C}}^\top\brak{\vec{B}+\vec{C}}}
    \end{align}
From the Equation\eqref{eq:Eqat2},We can do
\begin{align}
   &\implies \vec{A}^\top \brak{\vec{B}+\vec{C}}=\sqrt{\brak{\vec{B}+\vec{C}}^\top\brak{\vec{B}+\vec{C}}}\\
&\implies \vec{A}^\top \brak{\vec{B}+\vec{C}}=\sqrt{\norm{\vec{B}}^2+2\sbrak{\vec{B}^{\top}\vec{C}}+\norm{\vec{C}}^2}\\
&\implies \vec{A}^\top \brak{\vec{B}+\vec{C}}=\sqrt{{\vec{B}\vec{B}^{\top}}+2\sbrak{\vec{B}^{\top}\vec{C}}+{\vec{C}^\top\vec{C}}}\label{eq:Eqt6}
\end{align}
Substitute the $\vec{C}$ Value in the Equation\eqref{eq:Eqt6},We get
\begin{align}
&\implies\vec{A}^{\top}\brak{\vec{B}+\lambda\vec{e}_1+\vec{D}}=\sqrt{\vec{B}\vec{B}^{\top}+2\vec{B}^{\top}\brak{\lambda\vec{e}_1+\vec{D}}+\brak{\lambda\vec{e}_1+\vec{D}}^{\top}\brak{\lambda\vec{e}_1+\vec{D}}}\\
&\implies\lambda=\frac{\sqrt{\vec{B}\vec{B}^{\top}+2\brak{\vec{B}^{\top}\lambda\vec{e}_1+\vec{B}^{\top}\vec{D}}+\brak{\lambda\vec{e}_1+\vec{D}^{\top}}\brak{\lambda\vec{e}_1+\vec{D}}}-\vec{A}^\top\brak{\vec{B}+\vec{D}}}{\vec{A}^{\top}\vec{e}_1} \label{eq:final}
\end{align}
Substitute the Given Data in Equation\eqref{eq:final},
\begin{align*}
\vec{A}=\myvec{1\\1\\1};\vec{B}=\myvec{2\\4\\-5};\vec{C}=\myvec{\lambda\\2\\3}
\end{align*}
we get,
\begin{align}   
&\implies\lambda=\frac{\sqrt{45+2\brak{2\lambda-7}+\lambda^2+13}-6}{1}\\
&\implies\lambda=\sqrt{44+4\lambda+\lambda^2}-6\\
&\implies\lambda+6=\sqrt{44+4\lambda+\lambda^2}\\
&\implies\brak{\lambda+6}^{2}=44+4\lambda+\lambda^2\\
&\implies\lambda^2+36+12\lambda=44+4\lambda+\lambda^2\\
&\implies12\lambda-4\lambda=44-36\\
&\impliedby8\lambda=8\\
 &\implies \lambda = 1
\end{align}
\end{enumerate}
\end{document}
