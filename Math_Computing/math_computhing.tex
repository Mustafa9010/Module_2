\documentclass[11pt, a4paper]{article}
\usepackage{sectsty}
\usepackage{graphicx}
\usepackage{amsmath}
\usepackage{amssymb}
\usepackage{setspace}
\usepackage{tasks}
\usepackage{graphicx}
\usepackage{float}
\usepackage{comment}
\usepackage{listings}
\usepackage[utf8]{inputenc}
\usepackage{amsfonts}
\usepackage{gensymb}
\usepackage{multicol}
\usepackage{tabularx}
\usepackage{tikz}
\newcommand{\myvec}[1]{\ensuremath{\begin{pmatrix}#1\end{pmatrix}}}
\let\vec\mathbf
\newcommand{\abs}[1]{\left| #1 \right|}
\newcommand{\mydet}[1]{\ensuremath{\begin{vmatrix}#1\end{vmatrix}}}
\providecommand{\brak}[1]{\ensuremath{\left(#1\right)}}
\providecommand{\lbrak}[1]{\ensuremath{\left(#1\right.}}
\providecommand{\rbrak}[1]{\ensuremath{\left.#1\right)}}
\providecommand{\cbrak}[1]{\ensuremath{\left\{#1\right\}}}
\providecommand{\sbrak}[1]{\ensuremath{{}\left[#1\right]}}
\providecommand{\norm}[1]{\left\lVert#1\right\rVert}
\providecommand{\abs}[1]{\left\vert#1\right\vert}
\usepackage[left=1in, right=0.5in, top=1in, bottom=1in]{geometry}
\title{ Math computing}
\author{ Mustafa}
\date{\today}
\let\cleardoublepage\clearpage
\begin{document}
\section*{NCERT CLASS 12}
\subsection*{CHAPTER 10 : EXERCISE 5.13}
\begin{enumerate}
\item\textbf{}The scalar product of the vector $\hat{i}+\hat{j}+\hat{k}$ with a unit vector along the sum of vectors $2\hat{i}+4\hat{j}-5\hat{k}$ and $\lambda\hat{i}+2\hat{j}+3\hat{k}$ is equal to one, Find the value of $\lambda$.
\\\\
\textbf{Solution:}\\\\
From the given data,We get
\begin{align}
\vec{A} =\myvec{1\\1\\1} , \vec{B}=\myvec{2\\4\\-5} , \vec{C}=\myvec{\lambda\\2\\3}
\end{align}
We now that 
\begin{align}
\vec{A}^\top\frac{\brak{\vec{B}+\vec{C}}}{\norm{\vec{B}+\vec{C}}}=1 \label{eq:Eqt}\\
 \implies{\vec{A}^\top=\myvec{1&1&1}}
\end{align}
From that,
\begin{align}
  \implies  \brak{\vec{B}+\vec{C}}=\myvec{2\\4\\-5}+\myvec{\lambda\\2\\3}=\myvec{2+\lambda\\6\\-2}\\
  \implies  \norm{\vec{B}+\vec{C}}={\sqrt{\brak{\vec{B}+\vec{C}}^\top \brak{\vec{B}+\vec{C}}}}={\sqrt{\lambda^2+4\lambda+44}}
\end{align}

From the Equation \eqref{eq:Eqt} we get,
 \begin{align}        
 \vec{A}^\top\brak{\vec{B}+\vec{C}}={\norm{\vec{B}+\vec{C}}}\\        
        \myvec{1&1&1}\myvec{2+ \lambda \\ 6 \\ -2}= {\sqrt{\brak{ \lambda ^2+4 \lambda +44}}} \\
        \lambda +6 = \sqrt{\brak{\lambda ^2+4 \lambda +44}}\\
        \brak{ \lambda +6}^2 = \brak{\lambda} ^2+4 \brak{\lambda} +44\\
       \brak{\lambda}^2+12\brak{ \lambda }+36 = \brak{ \lambda }^2+4 \brak{\lambda} +44\\
     8\brak{\lambda} = 8\\
    \implies    \lambda = 1
 \end{align}
\end{enumerate}
\end{document}
