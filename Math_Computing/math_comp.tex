\documentclass[11pt, a4paper]{article}
\usepackage{sectsty}
\usepackage{graphicx}
\usepackage{amsmath}
\usepackage{amssymb}
\usepackage{setspace}
\usepackage{tasks}
\usepackage{graphicx}
\usepackage{float}
\usepackage{comment}
\usepackage{listings}
\usepackage[utf8]{inputenc}
\usepackage{amsfonts}
\usepackage{gensymb}
\usepackage{multicol}
\usepackage{tabularx}
\usepackage{tikz}
\newcommand{\myvec}[1]{\ensuremath{\begin{pmatrix}#1\end{pmatrix}}}
\let\vec\mathbf
\newcommand{\abs}[1]{\left| #1 \right|}
\newcommand{\mydet}[1]{\ensuremath{\begin{vmatrix}#1\end{vmatrix}}}
\providecommand{\brak}[1]{\ensuremath{\left(#1\right)}}
\providecommand{\lbrak}[1]{\ensuremath{\left(#1\right.}}
\providecommand{\rbrak}[1]{\ensuremath{\left.#1\right)}}
\providecommand{\cbrak}[1]{\ensuremath{\left\{#1\right\}}}
\providecommand{\sbrak}[1]{\ensuremath{{}\left[#1\right]}}
\providecommand{\norm}[1]{\left\lVert#1\right\rVert}
\providecommand{\abs}[1]{\left\vert#1\right\vert}
\usepackage[left=1in, right=0.5in, top=1in, bottom=1in]{geometry}
\title{ Math computing}
\author{ Mustafa}
\date{\today}
\let\cleardoublepage\clearpage
\begin{document}
\section*{NCERT CLASS 12}
\subsection*{CHAPTER 10 : EXERCISE 5.13}
\begin{enumerate}
\item\textbf{}The scalar product of the vector $\hat{i}+\hat{j}+\hat{k}$ with a unit vector along the sum of vectors $2\hat{i}+4\hat{j}-5\hat{k}$ and $\lambda\hat{i}+2\hat{j}+3\hat{k}$ is equal to one, Find the value of $\lambda$.
\\\\
\textbf{Generalized Construction:}\\
We now that \\
\begin{align}
   &\implies \vec{A}^\top = \frac{\brak{\vec{B}+\vec{C}}}{\norm{\vec{B}+\vec{C}}}\\
       &\implies \vec{A}^\top \brak{\vec{B}+\vec{C}}=\norm{\vec{B}+\vec{C}} \label{eq:Eqat2}
    \end{align}
    were,\\
    \begin{align}
        & \implies \vec{C}=\lambda e_1+\vec{D}\label{eq:Eqt1}
    \end{align}
Let us consider the L.H.S of Equation\eqref{eq:Eqat2},and we get $\vec{C}$ value from \eqref{eq:Eqt1}
\begin{align}
    &\implies \vec{A}^\top \brak{\vec{B}+\vec{C}}\\
   &\implies \vec{A}^\top \brak{\vec{B}+\lambda e_1+\vec{D}}
\end{align}
Now let us consider R.H.S of Equation\eqref{eq:Eqat2},we get,
\begin{align}
    &\implies \sqrt{\brak{\vec{B}+\vec{C}}^\top\brak{\vec{B}+\vec{C}}}
\end{align}
We get Final Generalized Equation
\begin{align}
   &\implies \vec{A}^\top \brak{\vec{B}+\lambda e_1+\vec{D}}=\sqrt{\brak{\vec{B}+\vec{C}}^\top\brak{\vec{B}+\vec{C}}}
   \label{eq:Eqa5}
\end{align}
Substitute the Given Data in Equation\eqref{eq:Eqa5},
\begin{align*}
\vec{A}=\myvec{1\\1\\1};\vec{B}=\myvec{2\\4\\-5};\vec{C}=\myvec{\lambda\\2\\3}
\end{align*}
we get,
\begin{align}   
&\implies 1\brak{2+\lambda}+1\brak{4+2}+1\brak{-5+3}=\sqrt{\brak{2+\lambda}^2+\brak{4+2}^2+\brak{-5+3}^2}\\
& \implies \lambda +6 = \sqrt{\brak{\lambda ^2+4 \lambda +44}}\\
& \implies\brak{ \lambda +6}^2 = \brak{\lambda} ^2+4 \brak{\lambda} +44\\
& \implies \brak{\lambda}^2+12\brak{ \lambda }+36 = \brak{ \lambda }^2+4 \brak{\lambda} +44\\
& \implies  8\brak{\lambda} = 8\\
 &\implies \lambda = 1
\end{align}
\end{enumerate}
\end{document}
