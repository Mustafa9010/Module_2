\documentclass[10pt]{book}
\usepackage{gvv-book}                                
\usepackage{gvv} 
\usepackage[sectionbib,authoryear]{natbib}% for name-date citation comment the below line
\usepackage{setspace} 
\setstretch{1.0}
\setcounter{secnumdepth}{3} 
\setcounter{tocdepth}{2}
\makeindex
\let\cleardoublepage\clearpage
\begin{document}
\frontmatter
%%%%%%%%%%%%%%%%%%%%%%%%%%%%%%%%%%%%%%%%%%%%%%%%%%%%%%%%%%%%%%%%
\booktitle{Geometry}
\subtitle{Through Algebra}
\AuAff{Mustafa Shaik}
%\halftitlepage
\titlepage
\tableofcontents
%\listoffigures %optional
%\listoftables  %optional
%% before \tableofcontents
%%%%%%%%%%%%%%%%%%%%%%%%%%%%%%%%%%%%%%%%%%%%%%%%%%%%%%%%%%%%%%%%
\setcounter{page}{0}
\begin{introduction}
This book shows how to solve problems in geometry using trigonometry and coordinate geometry.
\end{introduction}
\mainmatter
\chapter{Triangle}
Consider a triangle with vertices
                \begin{align}
                        \label{eq:tri-pts}
                        \vec{A} = \myvec{5 \\ -2},\,
                        \vec{B} = \myvec{-5 \\ 5},\,
                        \vec{C} = \myvec{-2 \\ -5}
                \end{align}
\section{Vectors}
\begin{enumerate}[label=\thesection.\arabic*.,ref=\thesection.\theenumi]
\numberwithin{equation}{enumi}
\item The direction vector of $\vec{AB}$ is defined as
\begin{align}
    \vec{B}-
    \vec{A}
\end{align} 
 Find the direction vectors of $\vec{AB}, \vec{BC}$ and $\vec{CA}$.\\
\solution\\
\begin{enumerate}
\item The direction vector of $\vec{AB}$ is
\begin{align}
    &= \vec{B}- \vec{A}\\ &= \myvec{ -5  - 5 \\  5  -(-2) }\\ &= \myvec{ -10 \\ 7}
    \label{eq:geo-dir-vec-ab}
\end{align}
\item The direction vector of $\vec{BC}$ is
\begin{align}
    &= \vec{C}- \vec{B}\\ &= \myvec{ -2 - (-5)\\ -5 - 3}\\ &=\myvec{ 3 \\ -10}
    \label{eq:geo-dir-vec-bc}
\end{align}
\item The direction vector of $\vec{CA}$ is
\begin{align}
    &= \vec{A}- \vec{C}\\ &= \myvec{ 5 - (-2)\\ -2 -(-5)}\\ &=\myvec{ 7 \\ 3}
    \label{eq:geo-dir-vec-ca}
\end{align}
\end{enumerate}
\item The length of side $\vec{AB}$ is
\begin{align}
\norm{\vec{B}-\vec{A}} \triangleq \sqrt{\brak{\vec{B}-\vec{A}}^{\top}{\vec{B}-\vec{A}}}
\end{align}\\
where,
\begin{align}
\begin{split}
\vec{A}^{\top}\triangleq\myvec{5 & -2}
\end{split}
\end{align}
\solution \\
Given,\\
                \begin{align}
                \vec{A} = \myvec{5 \\ -2},
                \vec{B} = \myvec{-5 \\ 5},
                \vec{C} = \myvec{-2 \\ -5}
                \end{align}
                \begin{align}
                \norm{\vec{B}-\vec{A}} \ &= \sqrt{\brak{\vec{B}-\vec{A}}^{\top}\brak{\vec{B}-\vec{A}}} \\
                \vec{B}-\vec{A} &= \myvec{-5 \\ 5} - \myvec{5 \\ -2} \\
                \vec{B}-\vec{A} &= \myvec{-10 \\ 7} \\
                \brak{\vec{B}-\vec{A}}^{\top} &= {\myvec{-10 \\ 7}}^{\top} = \myvec{-10 \ &7} \\
                \brak{\vec{B}-\vec{A}}^{\top}\brak{\vec{B}-\vec{A}} &= \myvec{-10 & 7} \myvec{-10 \\ 7} \\
                 & = 100 + 49 \\
                 & = 149 \\
                \sqrt{\brak{\vec{B}-\vec{A}}^{\top}\brak{\vec{B}-\vec{A}}} &= \sqrt{149} \\
                \label{eq:geo-norm-bc}
                 \implies \norm{\vec{B}-\vec{A}} &= \sqrt{149}
                \end{align}
Now solving for $\vec{BC}$,
\begin{align}
\vec{C}-\vec{B} &= \myvec{-3\\-10}\\
\norm{\vec{C}-\vec{B}} &= \sqrt{\myvec{3 & -10}\myvec{3\\-10}}\\
&= \sqrt{\brak{3}^2 +\brak{-10}^2}\\
&=\sqrt{109}
\label{eq:geo-norm-ab}
\end{align}
Now solving for $\vec{CA}$,
    \begin{align}  
    	\vec{A}-\vec{C} &= \myvec{7\\3}\\
    	\norm{\vec{A}-\vec{C}} &= \sqrt{\myvec{7 & 3}\myvec{7\\3}}\\
        &= \sqrt{\brak{7}^2+\brak{3}^2}\\ &=\sqrt{58}
        \label{eq:geo-norm-ca}
    \end{align}
\item   Points $\vec{A}, \vec{B}, \vec{C}$ are defined to be collinear if
\begin{align}
\rank{\myvec{1 & 1 & 1 \\ \vec{A}& \vec{B}&\vec{C}}} = 2
\end{align}
Are the given points in
\eqref{eq:tri-pts}
collinear?
\\          
Question : Check the collinearity of $\vec{A},\vec{B},\vec{C}$
\\
\solution
Given that,
\begin{align}
    \vec{A} = \myvec{5\\-2
    \quad
    \vec{B} &= \myvec{-5\\5}
    \quad
    \vec{C} = \myvec{-2\\-5}
\end{align}
Given that $\vec{A},\vec{B},\vec{C}$ are collinear if
\begin{align}
    \text{rank}\myvec{
    1 & 1 & 1\\
    \vec{A} & \vec{B} & \vec{C} \\
    } &< 3
    \label{eq:1.1.3,2}
\end{align}
Let
\begin{align}
    \vec{R}&=\myvec{
    1 & 1 & 1
    \\
    5 & -5 & -2
    \\
    -2 & 5 & -5
    }
\end{align}
The matrix $\vec{R}$ can be row reduced as follows,
\begin{align}
    \label{eq:matthrowoperations}
    \myvec{
    1 & 1 & 1
    \\
    5 & -5 & -2
    \\
    -2 & 5 & -5
    }
     \xleftrightarrow[]{R_2 \leftarrow R_2-5R_1}
    \myvec{ 1 & 1 & 1
    \\
    0 & -10 & -7
    \\
    -2 & 5 & -5
    }
    \\
     \xleftrightarrow[]{R_3\leftarrow R_3+2R_1}
    \myvec{
    1 & 1 & 1
    \\
    0 & -10 & -7
    \\ 0 & 7 & -3
    }
    \\
     \xleftrightarrow[]{R_3\leftarrow R_3+(7R_2/10)}
    \myvec{
    1 & 1 & 1
    \\
    0 & -10 & -7
    \\    
    0 & 0 & -79/10
    }
\end{align}
There are no zero rows. So,
\begin{align}
    \text{rank}\myvec{
    1 & 1 & 1\\
    \vec{A} & \vec{B} & \vec{C}\\
    } &= 3
\end{align}
Hence, from \eqref{eq:1.1.3,2} the points $\vec{A},\vec{B},\vec{C}$ are not collinear.
From Fig. \ref{fig1:Triangle}, We can see that $\vec{A},\vec{B},\vec{C}$ are not collinear .
\begin{figure}[H]
\centering
\includegraphics[width=\columnwidth]{figs/ABC_plot.png}
\caption{$\vec{A},\vec{B},\vec{C}$ plot}
\label{fig1:Triangle}
\end{figure}
\item The parameteric form of the equation  of $\vec{AB}$ is
                \begin{align}
                \label{eq:geo-param}
                \vec{x}=\vec{A}+k\vec{m}
                \end{align}
                where
                \begin{align}
		\vec{m}=\vec{B}-\vec{A}
                \end{align}
is the direction vector of $\vec{AB}$. Now Find the parameteric equations of $\vec{AB}, \vec{BC}$ and $\vec{CA}$.
\\  
\solution 
\begin{enumerate}
\item Parametric form of $\vec{AB}$ :
\begin{align}
\vec{x} = \vec{A} + k\vec{m}
\end{align}
where,
\begin{align}
\vec{m} = \vec{B} - \vec{A}
\end{align}
\begin{align}
\vec{B} - \vec{A} &= \myvec{-5\\5} - \myvec{ 5\\-2}\\
&= \myvec{-5-(5) \\ (5)-(-2)}\\
\implies \vec {m} &= \myvec{-10\\7}
\end{align}
therefore,
\begin{align}
\vec{AB} : \vec{x} &= \myvec{5\\-2} + k\myvec{-10\\7}
\end{align}
\item Parametric form of line $\vec{BC}$ :
\begin{align}
\vec{x} = \vec{B} + k\vec{m}
\end{align}
\begin{align}
\text{BC : } \vec{x} =& \myvec{-5\\5} + k\myvec{3\\-10}
\end{align}
\item Parametric form of line $\vec{CA}$ :
\begin{align}
\vec{x} = \vec{C}+ k\vec{m}
\end{align}
\begin{align}
\text{CA : } \vec{x} =& \myvec{-2\\-5} + k\myvec{7\\3}
\end{align}
\end{enumerate}
 \item The normal form of the equation of $\vec{AB}$ is
\begin{align}
\vec{n}^{\top}\myvec{\vec{x}-\vec{A}}=0
\end{align}
where
\begin{align}
\vec{n}^{\top}\vec{m}&=\vec{n}^{\top}\myvec{\vec{B}-\vec{A}}=0
\end{align} 
or,\begin{align}
\vec{n}&=\myvec{0 &1 \\-1 & 0}\vec{m}
\end{align}
Find the normal form of the equations of $\vec{AB}$, $\vec{BC}$ and $\vec{CA}$\\
\solution:\\
       The normal equation for the side $\vec{AB}$ is
\begin{align}
\vec{n}^{\top}\myvec{\vec{x}-\vec{A}}&=0\\
\implies
\vec{n}^{\top}\vec{x}&=\vec{n}^{\top}\vec{A}
\end{align}
Now our task is to find the $\vec{n}$ so that we can find $\vec{n}^{\top}$.
As given. 
\begin{align}
  \vec{n} &= \myvec{0 & 1\\
  -1 & 0}\vec{m}
\end{align}
Here, $\vec{m} = \vec{B}- \vec{A}$ for side $\vec{AB}$
\begin{align}
\implies
\vec{m}&=\myvec{-5\\5} - \myvec{5\\-2}\\
&=\myvec{-10\\7}
\end{align}
Now as we have obtained vector $\vec{m}$.we can use this to obtain vector $\vec{n}$
\begin{align}
\vec{n} &= \myvec{0 & 1\\
  -1 & 0}\myvec{-10\\7}
 = \myvec{7\\10}
\end{align}
The transpose of $\vec{n}$ is
\begin{align}
  \vec{n}^{\top}&=\myvec{7 & 10}
\end{align}
Hence the normal equation of side $\vec{AB}$ is 
\begin{align}
    \myvec{7 & 10}\vec{x}&=\myvec{7 & 10}\myvec{5\\-2}\\
    \implies \myvec{7 & 10}\vec{x} &= 15
\end{align}
\begin{figure}[H]
\includegraphics [width=\columnwidth]{figs/AB_line.png}
\caption{ The line $\vec{AB}$ plotted}
\label{fig:line AB}
\end{figure}
Similarly,
\begin{align}
	\implies
	\vec{BC:} \myvec{10 & -3}\vec{x} &= -35
\end{align}
\begin{figure}[H]
\includegraphics [width=\columnwidth]{figs/BC_line.png}
\caption{ The line $\vec{BC}$ plotted}
\label{fig:line BC}
\end{figure}
\begin{align}
	\implies \vec{CA:} \myvec{3 & -7}\vec{x} &= 29
\end{align}
\begin{figure}[H]
\includegraphics [width=\columnwidth]{figs/CA_line.png}
\caption{ The line $\vec{CA}$ plotted}
\label{fig:line CA}
\end{figure}
\item The area of $\triangle ABC$ is defined as
\begin{align}
\frac{1}{2}\norm{{\brak{\vec{A}-\vec{B}}\times {\brak{\vec{A}-\vec{C}}}}}
\end{align}
where,
\begin{align}
\vec{A}\times\vec{B} \triangleq \mydet{5 & -5 \\-2& 5}
\end{align}
Find the area of $\triangle ABC$.\\
\solution\\
Given,
\begin{align}
 \vec{A} = \myvec{5\\-2};
 \vec{B} = \myvec{-5\\5};
 \vec{C} = \myvec{-2\\-5}
 \end{align}
 \begin{align}
 \vec{A}-\vec{B} &= \myvec{5\\-2} - \myvec{-5\\5} = \myvec{10\\-7}\\
 \vec{A}-\vec{C} &= \myvec{5\\-2} - \myvec{-2\\-5} = \myvec{7\\3}\\
\therefore(\vec{A}-\vec{B})\times(\vec{A}-\vec{C}) 
 &= \mydet{10 & 7\\-7 & 3}\\
 &= \brak{10 \times 3} - \brak{-7 \times 7}\\ &= 30 + 49\\ &= 79\\
 \implies\frac{1}{2}\norm{(\vec{A}-\vec{B})\times(\vec{A}-\vec{C})} &= \frac{1}{2}\norm{79}= \frac{79}{2}
\end{align}
\item Find the angles $A, B, C$ if
    \label{prop:angle2d}
    \begin{align}
    \label{eq:angle2d}
    \cos A \triangleq
\frac{\brak{\vec{B}-\vec{A}}^{\top}\brak{{\vec{C}-\vec{A}}}}{\norm{\vec{B}-\vec{A}}\norm{\vec{C}-\vec{A}}}
\end{align}
\solution\\
From the given values of $\vec{A},\vec{B},\vec{C}$,\\
\begin{enumerate}
 \item Finding the value of angle $A$
\begin{align}
 \vec{B}-\vec{A} &=\myvec{-10\\7}
\end{align}
and 
\begin{align}
 \vec{C}-\vec{A} &= \myvec{-7\\-3}
\end{align}
also calculating the values of norms
\begin{align}
 \norm{\vec{B}-\vec{A}} &= \sqrt{149}\\
 \norm{\vec{C}-\vec{A}} &= \sqrt{58}
\end{align}
and by doing matrix multiplication we get,
\begin{align}
\begin{split}
 (\vec{B}-\vec{A})^{\top}(\vec{C}-\vec{A}) &= \myvec{-10&7}\myvec{-7\\-3} = 49
\end{split}
\end{align}
So, we get
\begin{align}
 \cos{A} &= \frac{49}{\sqrt{149} \sqrt{58}}\\
 &= \frac{49}{92.96}\\
 \implies A& = \cos^{-1}{\frac{49}{92.96}}
\end{align}
\item Finding the value of angle B
\begin{align}
 \vec{C}-\vec{B} &=\myvec{ 3\\-10}
\end{align}
and 
\begin{align}
 \vec{A}-\vec{B} &= \myvec{10\\-7}
\end{align}
also calculating the values of norms
\begin{align}
 \norm{\vec{C}-\vec{B}} &= \sqrt{109}\\
 \norm{\vec{A}-\vec{B}} &= \sqrt{149}
\end{align}
and by doing matrix multiplication we get,
\begin{align}
\begin{split}
 (\vec{C}-\vec{B})^{\top}(\vec{A}-\vec{B}) &= \myvec{3&-10}\myvec{10\\-7} = 100
\end{split}
\end{align}
So, we get 
\begin{align}
	\cos{B} &= \frac{100}{\sqrt{149} \sqrt{109}}\\
 &= \frac{100}{127.44}\\
 \implies B& = \cos^{-1}{\frac{100}{127.44}}
\end{align}

\item Finding the value of angle C
\begin{align}
 \vec{A}-\vec{C} &=\myvec{7\\3}
\end{align}
and 
\begin{align}
 \vec{B}-\vec{C} &= \myvec{-3\\10}
\end{align}
also calculating the values of norms
\begin{align}
 \norm{\vec{A}-\vec{C}} &= \sqrt{58}\\
	\norm{\vec{B}-\vec{C}} &= \sqrt{109}
\end{align}
and by doing matrix multiplication we get,
\begin{align}
\begin{split}
 (\vec{A}-\vec{C})^{\top}(\vec{B}-\vec{C}) &= \myvec{7&3}\myvec{-3\\10}\\
 &= 9
\end{split}
\end{align}
so, 
\begin{align}
\cos{C} &= \frac{9}{{\sqrt{58}} \sqrt{109}}\\
 &= \frac{9}{79.511}\\
\implies C &= \cos^{-1}{\frac{9}{79.511}}
\end{align}
\end{enumerate} 
All codes for this section are available at 
\begin{lstlisting}
       geometry/Triangle/Vectors/codes/Triangle_sides.py
\end{lstlisting}
\end{enumerate}
\backmatter
\appendix
\latexprintindex
\end{document}}
