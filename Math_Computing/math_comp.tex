\documentclass[11pt, a4paper]{article}
\usepackage{sectsty}
\usepackage{graphicx}
\usepackage{amsmath}
\usepackage{amssymb}
\usepackage{setspace}
\usepackage{tasks}
\usepackage{graphicx}
\usepackage{float}
\usepackage{comment}
\usepackage{listings}
\usepackage[utf8]{inputenc}
\usepackage{amsfonts}
\usepackage{gensymb}
\usepackage{multicol}
\usepackage{tabularx}
\usepackage{tikz}
\newcommand{\myvec}[1]{\ensuremath{\begin{pmatrix}#1\end{pmatrix}}}
\let\vec\mathbf
\newcommand{\abs}[1]{\left| #1 \right|}
\newcommand{\mydet}[1]{\ensuremath{\begin{vmatrix}#1\end{vmatrix}}}
\providecommand{\brak}[1]{\ensuremath{\left(#1\right)}}
\providecommand{\lbrak}[1]{\ensuremath{\left(#1\right.}}
\providecommand{\rbrak}[1]{\ensuremath{\left.#1\right)}}
\providecommand{\cbrak}[1]{\ensuremath{\left\{#1\right\}}}
\providecommand{\sbrak}[1]{\ensuremath{{}\left[#1\right]}}
\providecommand{\norm}[1]{\left\lVert#1\right\rVert}
\providecommand{\abs}[1]{\left\vert#1\right\vert}
\usepackage[left=1in, right=0.5in, top=1in, bottom=1in]{geometry}
\title{ Math computing}
\author{ Yaswanth }
\date{\today}
\let\cleardoublepage\clearpage
\begin{document}
\section*{NCERT CLASS 12}
\subsection*{CHAPTER 10 : EXERCISE 5.13}
\begin{enumerate}
\item The Scalar product of the Vector $(\hat{i}+\hat{j}+\hat{k})$ with a Unit Vector along the Sum of Vector $(2\hat{i}+4\hat{j}-5\hat{k})$ and $(\lambda \hat{i}+2\hat{j}+3\hat{k})$ is Equal to One.Find the Value of "$\lambda$". 
\\

\textbf{CONSTRUCTION STEPS :}
\begin{enumerate}
    \item Let us consider the three Different Vectors,
    \begin{align}
        a_1\hat{i}+b_1\hat{j}+c_1\hat{k} \label{eq:Equation 1}\\
                a_2\hat{i}+b_2\hat{j}+c_2\hat{k} \label{eq:Equation 2}\\
        a_3\hat{i}+b_3\hat{j}+c_3\hat{k} \label{eq:Equation3}
    \end{align}
     \item Let us assume,The Sum of the two Vectors, Equation\eqref{eq:Equation 2} and Equation\eqref{eq:Equation3},we get \begin{align}
         (a_2+a_3)\hat{i}+(b_2+b_3)\hat{j}+(c_2+c_3)\hat{k}\label{eq:Equation4}   
         \end{align}
     \item Unit Vector of Equation\eqref{eq:Equation4},we get
      \begin{align}
          \frac{(a_2+a_3)\hat{i}+(b_2+b_3)\hat{j}+(c_2+c_3)\hat{k}}{\sqrt{(a_2+a_3)^2+(b_2+b_3)^2+(c_2+c_3)^2}} \label{eq:Eqt5}
      \end{align}
      \item As we know that,Scalar Product of Equation\eqref{eq:Equation 1} and Equation\eqref{eq:Eqt5} is Equals to 1,we get
      \begin{align}
          (a_1\hat{i}+b_1\hat{j}+c_1\hat{k})\times\frac{(a_2+a_3)\hat{i}+(b_2+b_3)\hat{j}+(c_2+c_3)\hat{k}}{\sqrt{(a_2+a_3)^2+(b_2+b_3)^2+(c_2+c_3)\hat{k}}} = 1\label{eq:Eqt6}
      \end{align}
      \item The final Equation is 
      \begin{align}
          \frac{a_1(a_2+a_3)+b_1(b_2+b_3)+c_1(c_2+c_3)}{\sqrt{(a_2+a_3)^2+(b_2+b_3)^2+(c_2+c_3)^2}}\label{eq:Eqt7}
      \end{align}
\end{enumerate}
\textbf{INPUT MATRIX :}
\begin{align}
 \myvec{a_1&b_1&c_1\\a_2&b_2&c_2\\a_3&b_3&c_3}=\myvec{1 & 1 & 1\\ 2 & 4& -5 \\ \lambda &2 &3}\label{eq:Eq8}
\end{align}
\textbf{OUTPUT :}
    when the Input Matrix is substituted in Equation\eqref{eq:Eqt7},then we get
\begin{align}
    \frac{1(2+\lambda)+1(4+2)+1(-5+3)}{\sqrt{(2+\lambda)^2+(4+2)^2+(-5+3)^2}}=1\\
\frac{(2+\lambda)+6-2}{\sqrt{(2+\lambda)^2+36+4}}=1\\
\frac{(2+\lambda)+4}{\sqrt{(2+\lambda)^2+40}}=1\\
\frac{6+\lambda}{\sqrt{\lambda^2+4\lambda+44}}=1\\
\sqrt{\lambda^2+4\lambda+44}=6+\lambda\\
(6+\lambda)^2=\lambda^2+4\lambda+44\\
\lambda^2+36+12\lambda=\lambda^2+4\lambda+44\\
12\lambda-4\lambda=44-36\\
8\lambda=8\\
\lambda=1
\end{align}
\end{enumerate}
\end{document}

